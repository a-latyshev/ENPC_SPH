\documentclass{article}
\usepackage[utf8]{inputenc}

\newcommand\tens[1]{\underline{\underline{#1}}}
\newcommand\ve[1]{\underline{#1}}
\usepackage[utf8]{inputenc}     % for éô
\usepackage[a4paper, left=0.8in, right=0.8in, top=1in, bottom=1in]{geometry}
                                % for page size and margin settings
\usepackage{graphicx}           % for ?
\usepackage{float}
\usepackage{subfigure} 
\usepackage[squaren, Gray, cdot]{SIunits}
\usepackage{xspace}
\usepackage{array}
\usepackage{multirow}
\usepackage{amsmath,amssymb}    % for better equations
% \DeclareMathOperator{\commande}{texte}
\usepackage{amsthm}             % for better theorem styles
\usepackage{mathtools}          % for greek math symbol formatting
\usepackage{enumitem}           % for control of 'enumerate' numbering
\usepackage{listings}           % for control of 'itemize' spacing
\usepackage{todonotes}          % for clear TODO notes
\usepackage{newcent}            %pour la police
\usepackage{hyperref}           % page numbers and '\ref's become clickable
%%%%%%%%%%%%%%%%%%%%%%%

%%%%%%%%%%%%%%%%%%%%%%%
\newcommand\ul[1]{\underline{#1}}
\newcommand\uul[1]{\underline{\underline{#1}}}
\newcommand\uex{\ul{e}_x}
\newcommand\uey{\ul{e}_y}
%%%%%%%%%%%%%%%%%%%%%%%

\begin{document}
\begin{titlepage}
	\thispagestyle{empty}
	\newcommand{\HRule}{\rule{\linewidth}{0.5mm}}
	\center
	\textsc{\large ECOLE NATIONALE DES PONTS ET CHAUSSEES}\\[.7cm]
	\includegraphics[width=35mm]{img/ENPC_logo.png}\\[.5cm]
	\textsc{\large Departement Génie Mécanique et Matériaux - 2019/2020}\\[0.5cm]
	
	\vspace{2cm}
	
	\HRule \\[0.4cm]
	{\LARGE {\fontfamily{pag}\selectfont {SPH : Projet}}
    \vspace{0.4cm}
	\HRule \\[.5cm]

\vspace{3cm}

\large Réalisé par : 

\vspace{0.5cm}

{\fontfamily{pag}\selectfont{{\bfseries Andrey LATYSHEV} $\And$ {\bfseries Siyuan HE}}}
\\
\vspace{1cm}

% Encadré par : \bfseries\fontfamily{pag}\selectfont{Daniel Weisz-Patrault}

}
\end{titlepage}

% \newpage
% \vspace*{\stretch{1}}
% \begin{center}
%     \tableofcontents
% \end{center}
% \vspace*{\stretch{1}}

\newpage
\section{Implémentation d'un schéma de difusion de densité}

\section{Détection de surface libre}

\begin{equation}
	\uul{R}_i^{-1} = - \sum_{j} V_j \ul{\nabla} w_{ij} \otimes \ul{r}_{ij}
\end{equation}

\begin{align*}
	& \uul{R}_i^{-1} = - \sum_{j} \frac{m}{\rho_j} \ul{\nabla} w_{ij} \otimes \ul{r}_{ij} = - m \sum_{j} 
	\frac{| \ul{\nabla} w_{ij} |}{\rho_j |\ul{r}_{ij} |} \ul{r}_{ij} \otimes \ul{r}_{ij} = -m A \implies A = A^T \\
	& A = \sum_j \frac{| \ul{\nabla} w_{ij} |}{\rho_j |\ul{r}_{ij} |} \left[ (r_{ij}^x)^2 \uex \otimes \uex + (r_{ij}^y)^2 \uey \otimes \uey + (r_{ij}^x r_{ij}^y)^2 (\uex \otimes \uey + \uey \otimes \uex) \right] \\ 
	& \lambda_{\min} = \min\lambda(\uul{R}_i^{-1}) = \min\{ -\frac{m}{2} (\mathrm{tr} A \pm \sqrt{\mathrm{tr}^2 - 4 \det A}) \} = -\frac{m}{2} (\mathrm{tr} A + \sqrt{\mathrm{tr}^2 - 4 \det A})
\end{align*}

Pour réaliser cette fonctionnalité on a implémenté des variables de type des particules et la fonction \lstinline$particle_type$, qui retourne une list de deux éléments : la valeur propre minimale et le type de la particule considérée.  
% \\ \lstinline$LONELY_PARTICLE FREE_BOUNDARY_PARTICLE SURROUNDED_PARTICLE$
\lstset{language=matlab}   
\begin{lstlisting}           
	function particle_type = findParticleType(m,dwdr,rho_j,rPos)
    % R-1 = -m * A 
    global LONELY_PARTICLE FREE_BOUNDARY_PARTICLE SURROUNDED_PARTICLE;

    rNorm = sqrt(rPos(:,1).*rPos(:,1)+rPos(:,2).*rPos(:,2)); 
    a11_ij = dwdr.*rPos(:,1).*rPos(:,1)./(rho_j.*rNorm); 
    a12_ij = dwdr.*rPos(:,1).*rPos(:,2)./(rho_j.*rNorm); 
    a22_ij = dwdr.*rPos(:,2).*rPos(:,2)./(rho_j.*rNorm); 
    a11 = sum (a11_ij);
    a12 = sum (a12_ij);
    a22 = sum (a22_ij);
    trA = a11 + a22;
    detA = a11 * a22 - a12 * a12;
    lambda = -m * 0.5 * (trA + sqrt(trA * trA - 4 * detA));

    if lambda <= 0.2 
        particle_type = [lambda LONELY_PARTICLE];
    elseif (lambda > 0.2) && (lambda <= 0.75 ) 
        particle_type = [lambda FREE_BOUNDARY_PARTICLE];
    else 
        particle_type = [lambda SURROUNDED_PARTICLE];
    end
\end{lstlisting} 

\end{document}